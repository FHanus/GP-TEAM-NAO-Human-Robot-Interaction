\documentclass[conference]{IEEEtran}

\usepackage[utf8]{inputenc}
\usepackage[english]{babel}

\begin{document}

\title{Implementing 'Gordon': A NAO Robot Receptionist to Enhance Efficiency in Critical Services Using Choregraphe and Python}

\author{
\IEEEauthorblockN{
        Diana Milligan\IEEEauthorrefmark{1},
        Filip Hanuš\IEEEauthorrefmark{1},
        Harry Williams\IEEEauthorrefmark{1},\\
        Jack Thompson\IEEEauthorrefmark{1},
        Natalie Leung\IEEEauthorrefmark{1}
}
\IEEEauthorblockA{
        \IEEEauthorrefmark{1}School of Engineering, College of Art, Technology and Environment,\\ University of the West of England, Bristol, UK\\
        Email: \{diana2.milligan, filip2.hanus, harry4.williams, jack2.ould, wing7.leung\}@live.uwe.ac.uk}
}

\maketitle

\begin{abstract}

This paper presents what is in this abstract. 

\end{abstract}

\begin{IEEEkeywords}

NAO Robot, Human Robot Interaction.

\end{IEEEkeywords}

\section{Introduction}

The purpose of this project is to use a Nao robotic platform to effectively perform a task that requires direct interaction with human. 
There are many different metrics that can be used to quantify the success of such a system, but as a broader definition an effective system 
should be able to receive and convey relevant information to an untrained user to achieve a wider goal.

A task that lends itself to this project brief is a reception environment, particularly in a medical setting. According to 
(source), managerial staff shortages within the NHS has pushed clinical staff to spend more time on administrative task over 
patient-facing care. To effectively reduce staffing requirements and thus make the running of medical centres less resource intensive, robotic 
systems can be integrated to alleviate the bulk of repetitive tasks. A good low-risk opportunity for this kind of integration is within a 
receptionist role where any possible errors are of a significantly lower severity than other roles (e.g. medical diagnosis and treatment).
A study conducted by (source) tested the concept of a robotic receptionist for medical purposes, concluding that the robot displayed a 
"proffesional level of friendliness" that made it suited the role. However, the testing used a Wizard-of-Oz method with the robot only 
used as the interface between a user and a remote operator. Had all the behaviours been generated by the robot, there could have been 
a significant difference in how it was perceived by users.

Methods to create a sense of authority for robots has been explored through many different ways, many of which can be implemented within 
this project. (source) explored the effect of height on robotic telepresence systems, concluding that a shorter "leader" was less persuasive 
for a the human follower. As well as the robot's stature, (source) found that contextual information such as the location of an interaction 
helped users infer the robot’s role.

Appearance has been argued to be a less important factor by (source) who tested a variety of different variables for artificial agents, with an 
anthropomorphism and behaviour being " the most significant factors in predicting the trust and compliance with the robot" rather 
than physical appearance. A Nao platform has been used by (source) to test the limitations of such compliance, finding that users would 
refuse to follow challenging orders from a robot if factors such as perceived intelligence or professionalism were compromised.


\end{document}